\documentclass[12pt,a4paper]{article}
\usepackage{psfig,epsfig,graphicx,epstopdf}
\usepackage[UTF8]{ctex}
\usepackage{url}
\usepackage{IEEEtrantools}
  \usepackage{amsmath}
  \usepackage{amssymb}
  \usepackage{float}

% 由于论文一般比较长,为了方便查看,可以选择只编译 \includeonly{} 中的文件。
%\includeonly{tex/abstract,tex/chapter1,tex/chapter2,tex/appendix,tex/references,tex/toc}

\usepackage{indentfirst}
\usepackage[normal,footnotesize]{caption2}%%normal表示标题文本的格式是不足一行居中,多于左对齐;small是指标题号(如图..,表..)的字体大小。
\usepackage[sort]{cite}
\usepackage[top=3.9cm,bottom=3.5cm,left=2.8cm,right=3cm,headheight=1.2cm,footskip=0.8cm]{geometry}
\usepackage[colorlinks,linkcolor=black,anchorcolor=blue,citecolor=green]{hyperref}
\usepackage{pdfpages} % 插入pdf

%%%%%%%%%%%%%%%%%%%%%%%%
% 为了区分中文句号“。”和数学符号“o”,一般将其转换成英文句点“.”。
\catcode`\。=\active
\newcommand{。}{.}
%%%%%%%%%%%%%%%%%%%%%%%%

\makeatletter
\newcommand{\rmnum}[1]{\romannumeral #1}
\newcommand{\Rmnum}[1]{\expandafter\@slowromancap\romannumeral #1@}
\makeatother

%\textwidth=16.5cm
%\textheight=21.8cm
%\textwidth 14.3cm
%\textheight 19.9cm
%\oddsidemargin 1.2cm
%\oddsidemargin 1.2cm
%\evensidemargin 1.3cm
%\topmargin 0.3cm
%\headsep 25pt
%\hoffset=-28pt
%\voffset=-1.8cm
\def\baselinestretch{1.25}
\renewcommand{\theequation}%
  {\arabic{section}.\arabic{equation}}
  \renewcommand{\theequation}
  {\arabic{section}.\arabic{equation}}
\renewcommand{\thefigure}
  {\arabic{section}.\arabic{figure}}
\renewcommand{\thetable}
  {\arabic{section}.\arabic{table}}


\newcommand{\song}{\CJKfamily{song}} % 宋体 (Windows自带simsun.ttf)
\newcommand{\fs}{\CJKfamily{fs}} % 仿宋体 (Windows自带simfs.ttf)
\newcommand{\kai}{\CJKfamily{kai}} % 楷体 (Windows自带simkai.ttf)
\newcommand{\hei}{\CJKfamily{hei}} % 黑体 (Windows自带simhei.ttf)
\newcommand{\li}{\CJKfamily{li}} % 隶书 (Windows自带simli.ttf)
\newcommand{\you}{\CJKfamily{you}} % 幼圆 (Windows自带simyou.ttf)

\newcommand{\chuhao}{\fontsize{42pt}{\baselineskip}\selectfont} % 字号设置
\newcommand{\xiaochuhao}{\fontsize{36pt}{\baselineskip}\selectfont} % 字号设置
\newcommand{\yichu}{\fontsize{32pt}{\baselineskip}\selectfont} % 字号设置
\newcommand{\yihao}{\fontsize{26pt}{\baselineskip}\selectfont} % 字号设置
\newcommand{\erhao}{\fontsize{21pt}{\baselineskip}\selectfont} % 字号设置
\newcommand{\xiaoerhao}{\fontsize{18pt}{\baselineskip}\selectfont} % 字号设置
\newcommand{\sanhao}{\fontsize{15.75pt}{\baselineskip}\selectfont} % 字号设置
\newcommand{\sihao}{\fontsize{14pt}{\baselineskip}\selectfont} % 字号设置
\newcommand{\xiaosihao}{\fontsize{12pt}{\baselineskip}\selectfont} % 字号设置
\newcommand{\wuhao}{\fontsize{10.5pt}{\baselineskip}\selectfont} % 字号设置
\newcommand{\xiaowuhao}{\fontsize{9pt}{\baselineskip}\selectfont} % 字号设置
\newcommand{\liuhao}{\fontsize{7.875pt}{\baselineskip}\selectfont} % 字号设置
\newcommand{\qihao}{\fontsize{5.25pt}{\baselineskip}\selectfont} % 字号设置
%%%%%%%%%%%%%%%%%% 页眉与页脚 %%%%%%%%%%%%%%%%%%%%%%%%%%%%%%%%%%
\usepackage{fancyhdr,graphicx}
\newsavebox{\mygraphic}
\sbox{\mygraphic}{\includegraphics{logo.eps}}
\pagestyle{fancy}
\fancyhead{} % clear all header fields
\fancyhead[L]{\usebox{\mygraphic}}
\fancyfoot{} % clear all footer fields
\fancyfoot[C]{---\thepage---}
\renewcommand{\headrule}{%
\makebox[0pt][l]{\rule[0\baselineskip]{\headwidth}{0pt}}%
\rule[0\baselineskip]{\headwidth}{0pt} }

 %\renewcommand{\footrulewidth}{0.1pt}
 %\renewcommand{\headwidth}{\textwidth}


\begin{document}
\renewcommand{\refname}{\hei \sanhao 参考文献}
%%%%%%%%%%%%%%%%%%%%%%%%%%%%%%%%%%%%%%%%%%%%%%%%%%%%%%%%%%%%%%
%\
\includepdf{ccnu-cover.pdf} % 华师论文封面,在 ccnu-cover.doc 中修改,然后另存为 pdf 格式,替换掉模板
% 以下部分在 tex 文件夹中修改添加
% 标题页与授权页若不方便使用电子签名,可以先打印出来,签名后,扫描成 pdf 文件,按上述方法插入 tex 中
%%%%%%%%%%%%%%%%%%%% Chinese Cover %%%%%%%%%%%%%%%%%%%%%%%%%%%%%%%%%%%%%%%%%%%%
%\begingroup
%fancyfoot{} % clear all footer fields
%\thispagestyle{empty} %\markleft{\usebox{\mygraphic}}
\fancyfoot{}
\renewcommand{\thepage}{}
\begin{center}

%\hbox{} \vspace*{\fill}
\vspace*{2cm}
{\hei\yihao{{{\fontsize{28pt}{\baselineskip}\selectfont 硕士学位论文}}}}\\

\vspace*{2cm}

{\erhao \bf \hei 文章标题}\\


\vspace*{2.5cm}

\begin{huge}
\begin{center}
\begin{tabular}{cl}
  \xiaoerhao{\bf \hei 论文作者:}& \xiaoerhao{姓名} \\
  \xiaoerhao{\bf \hei 指导教师:}& \xiaoerhao{导师姓名\ 职称} \\
  \xiaoerhao{\bf \hei 学科专业:}& \xiaoerhao{专业名称} \\
  \xiaoerhao{\bf \hei 研究方向:}& \xiaoerhao{研究方向} \\
\end{tabular}
\end{center}
\end{huge}

\vspace*{3.5cm}

{\hei\xiaoerhao 华中师范大学物理科学与技术学院}

\vspace*{.5cm}

{\hei\xiaoerhao 2020年12月(修改时间)}

\vspace{\fill}
\end{center}

\newpage
%\endgroup

%%%%%%%%%%%%%%%%%%%% Second Cover %%%%%%%%%%%%%%%%%%%%%%%%%%%%%%%%%%%%%%%%%%%%
\newpage
%\begingroup
 \fancyfoot{} % clear all footer fields
%\thispagestyle{empty} %取消当前页码
\renewcommand{\thepage}{cover.en}
\begin{center}
%\hbox{} %\vspace*{\fill}

 \textbf{\xiaoerhao\bf \hei  论文英文标题}\\

 \vspace*{1.5cm}

{\large \sl A Thesis\\
\vspace*{0.4cm}
Submitted in Partial Fulfillment of the Requirements \\
\vspace*{0.4cm}For the Master's Degree in Astrophysics\\}

\vspace*{2cm}

{\normalsize\sanhao\hei\bf By}

\vspace*{0.4cm}

{\normalsize\sanhao\hei\bf 英文名}

\vspace*{0.4cm}

{\large \you \tt Postgraduate Program\\
\vspace*{0.4cm}
School of Physics and Technology\\
\vspace*{0.4cm} Central China Normal University\\}

\vspace*{2cm}

\textbf{\normalsize\sanhao\hei\bf Supervisor:
\hspace{1em}{\normalsize\sanhao\hei\bf  导师英文名}\hfill}

\vspace*{0.4cm}

\textbf{\normalsize\sanhao\hei\bf Academic Titles:
\hspace{1em}{\normalsize\sanhao\hei\bf 导师职称} \hfill}


\vspace*{0.5cm}

{\normalsize\sanhao \hfill Signature$\_ \_ \_ \_ \_ \_ \_ \_ \_ \_ \_ \_$}

\vspace*{0.3cm}

{\normalsize\sanhao \hfill Approved}

\vspace*{0.3cm}

{\normalsize\sanhao \hfill December, \hspace{0.5em}2020(修改时间)}

\vspace{\fill}

\end{center}
\newpage
%\endgroup
 % 标题页
\include{tex/authorization} % 授权页
\pagenumbering{roman}
\newpage
 \vspace{1.0cm}
 \headsep=1.5cm
 \centerline{\hei \sanhao 摘\ \ \ \ \ 要}
 \label{zhaiyao}
\vspace{0.5cm}
 \pagestyle{fancy}
 \fancyfoot[C]{---\thepage---}

这是华中师范大学物理科学与技术学院\textbf{非官方}硕士学位论文 \LaTeX 模板(官方只有 Word 模板),
理论上适用于所有专业。
原作者未知,若有人知晓,还望告知,我将正式征求授权和为其署名。

考虑到有人可能不熟悉 \LaTeX,
以及模板的流传范围太过狭窄,
我对其稍加修改,
私自上传到~\href{https://github.com/ichenh/ccnu-latex}{GitHub}~网站。
其中,我补充了更详细的说明,
添加了中文句号转换成英文句点命令,
增加了华师官方要求的论文(存档)封面,
将过期的 Hua-Zhong Normal University 图标换成了新的 Central China Normal University,
以及展示了一些基本的 \LaTeX 范例。
更详细的 \LaTeX 命令可参考 \CTeX 开发小组翻译的《\href{http://mirrors.ibiblio.org/CTAN/info/lshort/chinese/lshort-zh-cn.pdf}{一份(不太)简短的 \LaTeX $\ 2\varepsilon$介绍}》,
我将电子版放在了补充材料文件夹中,
其中也附上了华师官方的学位论文写作要求。
请随意使用和修改模板,欢迎补充,如有错漏,
还请在 GitHub 上提交 issue 或 Pull Request。

另外,建议使用 \href{https://ctan.org/mirrors/mirmon#cn}{TeX Live} 编译软件以及它自带的 TeXworks 编辑软件。
具体步骤是:

\begin{itemize}
\item[1.] 修改论文封面:编辑根目录中的 \verb!ccnu-cover.doc! 文件,另存为 \verb!ccnu-cover.pdf!,覆盖原有文件。
\item[2.] 打开 \verb!main.tex! 文件,按如下顺序进行排版生成 \verb!main.pdf! 文件:XeLaTeX $\to$ BibTeX $\to$ XeLaTeX $\to$ XeLaTeX。
\item[3.] 论文写作:打开 tex 文件夹,修改相应文件(对应于论文的不同部分和章节。
若要增加新的章节,复制一份 \verb!chapter2.tex!,修改成 \verb!chapter3.tex!,
在 \verb!main.tex! 中的 \verb!\include{tex/chapter3}! 下方添加 \verb!\include{tex/chapter3}!。以此类推。
\item[4.] 重复第二步,生成新的 \verb!main.pdf! 文件。
\end{itemize}

整体思路很简单,
由于论文一般很长,放在一个文件里不好整理,
所以分成多个文件,最后导入同一个文件中编辑。
根目录中的 \verb!main.tex! 就是主要文件,只需编译它即可。
至于目录、引言、正文、参考文献等内容则分别放在 tex 目录下,
然后在 \verb!main.tex! 文件中通过 \verb!\include{}! 命令导入。

第二步中的 XeLaTeX 负责编译中文文本,BibTeX 是编译 \verb!.bib! 格式的参考文献。
若是使用了论文管理软件,可以选择导出为 \verb!.BibTeX! 文件,
覆盖 tex 文件夹下的 \verb!references.bib! 文件。
或者自己搜索相关文献的 BibTeX 引用格式,
将其复制到 tex 目录下的 \verb!references.bib! 文件中。
通过 BibTeX 编译出参考文献之后就只需要选择 XeLaTeX 编译,除非增加了新的文献。

\vspace{1cm} \noindent {\sihao \hei \textbf{关键词:}}华中师范大学;物理科学与技术学院;硕博士论文模板



\newpage %\vspace{1.5cm}
\headsep=0.7cm
\centerline{\bf \sanhao Abstract}
\label{abs}
\vspace{0.5cm}

Emmm...

\vspace{1cm} \noindent {\sihao \textbf{Keywords: }}CCNU; College of Physical Science and Technology; Tempolate of Doctoral or Master thesis % 摘要
\newpage
\pagestyle{fancy}
\thispagestyle{empty}
 %\fancyfoot{}
 \vspace{1.0cm}
 \centerline{\hei \sanhao 目\ \ \ \ \ 录}
\vspace{0.5cm} \label{content} \noindent
{\hei \xiaosihao 摘要 \dotfill \pageref{zhaiyao}}\\[0.2cm]
{\hei \xiaosihao ABSTRACT \dotfill \pageref{abs}}\\[0.2cm]
%{\hei \xiaosihao 目录\dotfill \pageref{content}}\\[0.2cm]
{\hei \xiaosihao 第一章\ \ \ \ 引言 \dotfill \pageref{1}}\\[0.2cm]
\hspace*{0.5cm} \ \  \ 1.1\ \  第一节\dotfill \pageref{1.1}\\
\hspace*{0.5cm} \ \  \ 1.2\ \  第二节\dotfill \pageref{1.2}\\[0.2cm] % 每章末尾添加 [0.2cm]
{\hei \xiaosihao 第二章\ \ \ \ \LaTeX 举例 \dotfill \pageref{2}}\\[0.2cm]
\hspace*{0.5cm} \ \  \ 2.1\ \  宇宙学标准模型\dotfill \pageref{2.1}\\
\hspace*{0.5cm} \ \  \ 2.2\ \  粒子物理标准模型\dotfill \pageref{2.2}\\[0.2cm]
{\hei \xiaosihao 第三章\ \ \ \ 总结与展望 \dotfill \pageref{3}}\\[0.2cm]
{\hei \xiaosihao 附录A\ \ \ \ 附录A \dotfill \pageref{A}}\\
\hspace*{0.5cm} \ \  \ A.1\ 通过 GitHub 写论文\dotfill \pageref{A.1}\\[0.2cm]
{\hei \xiaosihao 参考文献 \dotfill \pageref{bib}}\\[0.2cm]
{\hei \xiaosihao 在校期间发表的论文、科研成果等\dotfill \pageref{work}} \\[0.2cm]
{\hei \xiaosihao 致谢\dotfill \pageref{acknowlegement}} % 目录,需手动输入章节名和序号
\pagenumbering{arabic}

\newpage
\vspace{1.0cm}
\setcounter{section}{1}
\setcounter{equation}{0}
\setcounter{figure}{0}
\setcounter{table}{0}
\section*{\centerline {\hei \sanhao  第一章 \ \ \ \ 引言}}
\label{1}
\vspace{0.5cm}

要开始写论文了!

\subsection{\sihao \hei 第一节}
\label{1.1}

好累啊,先看个泡面番休息一下吧。

\subsection{\sihao \hei 第二节}
\label{1.2}

咦,到饭点了。 % 第一章 引言
% 注意更改序号!请自行验证。
\newpage
\vspace{1.0cm}
\setcounter{section}{2} % 章节序号
\setcounter{subsection}{0} % 小节序号
\setcounter{equation}{0} % 公式序号
\setcounter{figure}{0} % 图片序号
\setcounter{table}{0} % 表格序号
\section*{\centerline {\hei \sanhao  第二章 \ \ \ \ \LaTeX 举例}}
\label{2} % 引用标签
\vspace{0.5cm}


\subsection{\sihao \hei 宇宙学标准模型}
\label{2.1}

以下介绍广义相对论~\cite{Einstein:1915ca, Einstein:1915by}~及其在宇宙学中的应用。%引用文献

从爱因斯坦-希尔伯特作用量出发:
% 行间公式
\begin{align}
S=\frac{1}{16\pi G}\int d^{4}x\sqrt{-g}\left(R-2\Lambda\right)+\int d^{4}x\sqrt{-g}\mathcal{L}_{\mathrm{Matter}}\,, 
\label{EH}
\end{align}
其中,$G$表示引力常数,$R$是里奇标量,$\Lambda$是宇宙学常数,$\mathcal{L}_{\mathrm{Matter}}$是物质的拉式量。

BLABLA...

于是从作用量~\eqref{EH}~%引用公式
我们可以得到爱因斯坦场方程:
\begin{align}
R_{\mu\nu}-\frac{1}{2}Rg_{\mu\nu}+\Lambda g_{\mu\nu}=8\pi GT_{\mu\nu}\,.
\label{GR}
\end{align}
其协变导数表明它满足能量守恒定律$\nabla^{\mu}T_{\mu\nu}=0$。%行内公式

\subsection{\sihao \hei 粒子物理标准模型}
\label{2.2}

\begin{figure}[!htbp]
\centering
\includegraphics[scale=0.7]{figures/trace.pdf}% 图片在 Figures 文件夹中,注意修改尺寸大小 scale
\caption{能动量张量在宇宙早期随温度的演化。}% 图例
\label{trace}
\end{figure}

其中,
\begin{align}
    \rho-3 p=\frac{g T^{4}}{2 \pi^{2}} \cdot x^{2} \int_{0}^{\infty} d y \frac{y^{2}}{\sqrt{x^{2}+y^{2}}} \frac{1}{\exp\left(\sqrt{x^{2}+y^{2}}\right) \pm 1}
    \,. % 标点符号
    \notag % 不添加序号
\end{align} % 第二章 正文,若要增加其他章节,复制一份修改即可,注意更改章节序号。
%\include{tex/chap3} % 第三章,以此类推
\newpage
 \vspace{1.0cm}
\setcounter{section}{3}
\setcounter{subsection}{0}
\setcounter{equation}{0}
\setcounter{figure}{0}
\setcounter{table}{0}
\section*{\centerline {\hei \sanhao  第三章 \ \ \ \ 总结与展望}}
\label{3}
\vspace{0.5cm}

啊,终于快写完了! % 总结与展望,这里为第三章

\newpage
 \vspace{1.0cm}
 \vspace{0.5cm}
\setcounter{section}{4}
\renewcommand{\theequation}{A.\arabic{equation}}
\setcounter{subsection}{0}
\setcounter{equation}{0}
\setcounter{figure}{0}
\setcounter{table}{0}
\section*{\centerline {\hei \sanhao  附录A\quad 附录标题}}
\label{A}

主线任务已达成,在附录中对相关内容做进一步补充和讨论吧。(字数还不够附录凑……)

\subsection*{\sihao \hei A.1\quad 通过 GitHub 写论文}
\label{A.1}

若是了解GitHub,那么建议在 GitHub 上建立一个私人仓库,
对论文进行版本控制。
具体操作方式,请自行搜索。 % 附录

\newpage
\label{bib}
\renewcommand\refname{参考文献}
\setlength{\baselineskip}{14pt}
\bibliographystyle{unsrt}
\bibliography{tex/references} % 注意是 BibTeX 格式,文件名为 references.bib

\newpage
\newpage
\vspace{1.0cm}
\section*{\centerline {\hei \sanhao  在校期间发表的论文、科研成果等}}
\label{work}
%\setlength{\baselineskip}{19pt}  \centerline{\hei \sanhao 在校期间发表的论文、科研成果等}
\vspace{0.5cm}

\begin{enumerate}
\item 在这里列举已发表的文章。若是盲审的论文,则只需要说明在什么期刊发表了多少篇文章。
\end{enumerate} % 在校期间发表的论文、科研成果等
\newpage
\newpage
\vspace{1.0cm}
\section*{\centerline {\hei \sanhao 致\quad 谢}}
\label{acknowlegement}
\vspace{0.5cm}

首先,感谢不知其名的原作者制作的这份模板,令我顺利地完成了硕士论文的写作。若有人知晓原作者的个人信息还请告知,我会正式征求他的同意,并署名。

其次,感谢岁月静好的研究生生活和所有温暖的人们。

最后,祝各位毕业生答辩顺利,前程似锦。


\rightline{}

\rightline{陈\ \ \ 华}

\rightline{2020年12月28日}

\rightline{于温暖的图书馆} % 致谢

\end{document}