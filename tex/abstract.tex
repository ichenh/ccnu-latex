\newpage
 \vspace{1.0cm}
 \headsep=1.5cm
 \centerline{\hei \sanhao 摘\ \ \ \ \ 要}
 \label{zhaiyao}
\vspace{0.5cm}
 \pagestyle{fancy}
 \fancyfoot[C]{---\thepage---}

这是华中师范大学物理科学与技术学院\textbf{非官方}硕士学位论文 \LaTeX 模板(官方只有 Word 模板),
理论上适用于所有专业。
原作者未知,若有人知晓,还望告知,我将正式征求授权和为其署名。

考虑到有人可能不熟悉 \LaTeX,
以及模板的流传范围太过狭窄,
我对其稍加修改,
私自上传到~\href{https://github.com/ichenh/ccnu-latex}{GitHub}~网站。
其中,我补充了更详细的说明,
添加了中文句号转换成英文句点命令,
增加了华师官方要求的论文(存档)封面,
将过期的 Hua-Zhong Normal University 图标换成了新的 Central China Normal University,
以及展示了一些基本的 \LaTeX 范例。
更详细的 \LaTeX 命令可参考 \CTeX 开发小组翻译的《\href{http://mirrors.ibiblio.org/CTAN/info/lshort/chinese/lshort-zh-cn.pdf}{一份(不太)简短的 \LaTeX $\ 2\varepsilon$介绍}》,
我将电子版放在了补充材料文件夹中,
其中也附上了华师官方的学位论文写作要求。
请随意使用和修改模板,欢迎补充,如有错漏,
还请在 GitHub 上提交 issue 或 Pull Request。

另外,建议使用 \href{https://ctan.org/mirrors/mirmon#cn}{TeX Live} 编译软件以及它自带的 TeXworks 编辑软件。
具体步骤是:

\begin{itemize}
\item[1.] 修改论文封面:编辑根目录中的 \verb!ccnu-cover.doc! 文件,另存为 \verb!ccnu-cover.pdf!,覆盖原有文件。
\item[2.] 打开 \verb!main.tex! 文件,按如下顺序进行排版生成 \verb!main.pdf! 文件:XeLaTeX $\to$ BibTeX $\to$ XeLaTeX $\to$ XeLaTeX。
\item[3.] 论文写作:打开 tex 文件夹,修改相应文件(对应于论文的不同部分和章节。
若要增加新的章节,复制一份 \verb!chapter2.tex!,修改成 \verb!chapter3.tex!,
在 \verb!main.tex! 中的 \verb!\include{tex/chapter3}! 下方添加 \verb!\include{tex/chapter3}!。以此类推。
\item[4.] 重复第二步,生成新的 \verb!main.pdf! 文件。
\end{itemize}

整体思路很简单,
由于论文一般很长,放在一个文件里不好整理,
所以分成多个文件,最后导入同一个文件中编辑。
根目录中的 \verb!main.tex! 就是主要文件,只需编译它即可。
至于目录、引言、正文、参考文献等内容则分别放在 tex 目录下,
然后在 \verb!main.tex! 文件中通过 \verb!\include{}! 命令导入。

第二步中的 XeLaTeX 负责编译中文文本,BibTeX 是编译 \verb!.bib! 格式的参考文献。
若是使用了论文管理软件,可以选择导出为 \verb!.BibTeX! 文件,
覆盖 tex 文件夹下的 \verb!references.bib! 文件。
或者自己搜索相关文献的 BibTeX 引用格式,
将其复制到 tex 目录下的 \verb!references.bib! 文件中。
通过 BibTeX 编译出参考文献之后就只需要选择 XeLaTeX 编译,除非增加了新的文献。

\vspace{1cm} \noindent {\sihao \hei \textbf{关键词:}}华中师范大学;物理科学与技术学院;硕博士论文模板



\newpage %\vspace{1.5cm}
\headsep=0.7cm
\centerline{\bf \sanhao Abstract}
\label{abs}
\vspace{0.5cm}

Emmm...

\vspace{1cm} \noindent {\sihao \textbf{Keywords: }}CCNU; College of Physical Science and Technology; Tempolate of Doctoral or Master thesis